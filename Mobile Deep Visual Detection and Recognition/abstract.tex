Deep learning wordt steeds meer gebruikt om beeldverwerkingsproblemen zoals \textbf{herkenningssystemen} en \textbf{detectiesystemen} op te lossen. 
Via neurale netwerken kunnen we met meer en betere features werken om de afbeeldingen te analyseren. 
Veel van deze modellen hebben echter behoorlijk wat rekenkracht en geheugen nodig om te werken.
Om deze reden wordt bij veel huidige mobiele toepassingen het deep learning model uitgevoerd in de cloud.
Hierdoor kan de mobiele toepassing echter niet werken zonder een internetconnectie.
Een ander probleem is dat veel operaties die een bestaand model uitvoert niet compatibel zijn met de \textbf{mobiele omgeving} waar we het model willen gebruiken.
Een veel gebruikte taal voor het ontwerpen en trainen van neurale netwerken is Python.
Deze modellen worden in een Python omgeving uitgevoerd terwijl Android studio applicaties in een Java omgeving worden uitgevoerd.
Het doel van deze masterproef is het onderzoeken van de \textbf{compatibiliteit} tussen de operaties van een \textbf{bestaand neuraal netwerk} in zijn originele omgeving en een mobiele omgeving, zodat een bestaand en complex deep learning model ge\"implementeerd kan worden op een mobiel apparaat met zo min mogelijk effect op de accuraatheid. 

We gaan deze compatibiliteit onderzoeken voor de ResNet50, YOLO en Faster-RCNN architectuur.
We vertrekken van bestaande architecturen die ontworpen en getraind zijn in het TensorFlow en PyTorch framework.
Deze twee frameworks hebben elk een eigen methode om het model te converteren naar een taalonafhankelijk model.
We stellen vast dat TensorFlow de meeste operaties ondersteund voor een Android studio implementatie.
De PyTorch bibliotheek voor Android studio bevat minder operaties dan TensorFlow maar voert zijn operaties wel sneller uit.
PyTorch moet zijn niet ondersteunde operaties op een alternatieve manier importeren in Android studio.
Uiteindelijk zijn de drie architecturen implementeerbaar in Android studio met weinig of geen effect op de accuraatheid.

\newpage
Deep learning is used more and more for vision applications like \textbf{image recognition} and \textbf{object detection}. 
With neural networks we are able to extract more and beter features from images.
Many of these neural networks require a lot of memory and computing power to extract those features.
Therefore, many mobile applications exists where the neural network is executed in the cloud.
In addition, by developing deep learning applications this way an internet connection is always required.
Furthermore, the operations of an existing neural network are often not compatibel with the \textbf{mobile environment} where the application is executed.
A coding language often used for designing and training neural networks is Python.
Consequently these neural networks are designed to be executed in a Python environment, while Android studio applications are executed in a Java environment.
The purpose of this thesis is to research the \textbf{compatibility} between operations of an \textbf{existing neural network} in its original environment and a mobile environment.
So we can implement complex neural networks on a mobile device with a minimal effect on its accuracy.
We will study this implementation for recognition systems and detection systems.

Futhermore we will study the compatibility for the ResNet50, Faster-RCNN and the YOLO architecture.
Initially we will study existing neural networks designed and trained in the TensorFlow and PyTorch framework.
Each framework has its own method to convert a neural network to a language independent version.
Thereafter we demonstrate that TensorFlow supports the most operations for an Android studio implementation.
While PyTorch supports less operations, it executes the supported operations faster.
PyTorch has to import its not supported operations with an alternate method in Android studio.
Eventuay al three architectures are executable in Android studio with a minimal effect on the accuracy.
%\textbf{Keywords}: Voeg een vijftal keywords in (bv: Latex-template, thesis, ...)
