%%%%%%%%%%%%%%%%%%%%%%%%%%%%%%%%%%%%%%%%%%%%%%%%%%%%%%%%%%%%%%%%%%% 
%                                                                 %
%                            CHAPTER                              %
%                                                                 %
%%%%%%%%%%%%%%%%%%%%%%%%%%%%%%%%%%%%%%%%%%%%%%%%%%%%%%%%%%%%%%%%%%% 

%\chapter{Vormelijke richtlijnen van de scriptie}
\chapter{Situering en doelstelling}

\section{Situering}
Tegenwoordig wordt deep learning steeds meer en meer gebruikt om beeldverwerking problemen op te lossen. Via neurale netwerken kunnen we met meer en betere features werken om de afbeeldingen te analiseren. Maar veel van deze modelen hebben behoorlijk wat rekenkracht en geheugen nodig om te werken. Ook is er steeds meer intresse naar real-time toepassingen waarvan het resultaat zo snel mogelijk beschikbaar moet zijn. Dit wordt moeilijk bij veel hedendaagse systemen waarbij de foto eerst genomen moet worden en vervolgens door een computer geanaliseerd moet worden, omdat hedendaagse systemen veel rekenwerk en geheugen vragen. In deze masterproef wordt er onderzocht of de computer kan weggelaten worden en de afbeelding meteen door het mobiel apparaat geanaliseerd kan worden. Dus er moet onderzocht worden hoe een bestaand model aangepast kan worden om effici\"ent te werken op een mobiel apparaat. Hierbij moet vooral rekening gehouden worden met de rekenkracht en geheugen van het mobiele apparaat.

\section{Probleemstelling}
Mobiele apparaten zijn kleine toestellen met beperkt geheugen en beperkte rekenkracht. In deze masterproef wordt er onderzocht hoe het rekenwerk beperkt kan worden zodat het resultaat real-time geleverd kan worden. Er gaat ook onderzocht worden hoe alle data effici\"ent kan worden opgeslagen op het toestel. 

\section{Doelstellingen}
Het uiteindelijke doel van deze masterproef is er voor zorgen dat een bestaand deep learning model aangepast kan worden zodat dit real-time resultaten kan geven op een mobiel apparaat. Dit gebeurt aan de hand van de volgende stappen:
\begin{itemize}
    \item grondig begrijpen van een deep learning herkenningssysteem
    \item grondig begrijpen van een deep learning detectiesysteem
    \item Onderzoeken welke technieken er gebruikt kunnen worden om bestaande systemen op een mobiel apparaat te implementeren.
    \item onderzoeken voor optimalisaties voor een herkenningssysteem
    \item onderzoeken voor optimalisaties voor een detectiesysteem
    \item gevonden technieken testen en analiseren
    \item werkend prototype applicatie ontwerpen voor een mobiel apparaat
\end{itemize}