\chapter{Compatibiliteit van detectie systemen}
Voor detectiesysteem bestuderen we uitgebreid de Faster-RCNN architectuur met een ResNet50 backbone en de YOLO architectuur.
We gaan voor deze modellen vertrekken vanuit het PyTorch en TensorFlow framework.
Om vervolgens de mogelijke paden te bestuderen naar een mobiele implementatie.

\section{faster rcnn}
met een resnet50 backbone

\section{Van TensorFlow naar mobiel mobiele implementatie}
Om van een TensorFlow object detectie model te gaan naar een mobiele implementatie kunnen we een TFlite of een ONNX model implementeren in Android Studio.
Het model dat we willen converteren is een voorgetrained Faster-RCNN object detectie model van de TensorFlow object detection API.
Dit Faster-RCNN model gaan we vervolgens converteren naar een TFLite of ONNX formaat.

\subsection{TFLite implementatie}

De TensorFlow object detection api stelt zelf een manier voor om een model te converteren naar het TFLite formaat.
Via de volgende lijnen code kunnen we een model converteren.
.....
Tijdens het converteren krijgen we de volgende error: ValueError: Only ssd or center\_net models are supported in tflite. Found faster\_rcnn in config.
Dit komt doordat de TFLite converter van de tensorflow object detection api enkel de conversie voor SSD en Centernet ondersteund.

Voor een object detectie model te converteren kan er ook gebruik gemaakt worden van de standaard TFLiteConverter.
Bij het converteren naar TFLite kan de ConcatV2 opperatie niet geconverteerd worden naar de TFLite concatenation opperatie.
%in het huidig model zijn er 2 gevallen waarbij de tf.ConcatV2 niet wordt geconverteerd naar tfl.concatenation.
%wat deze 2 gevallen van tf.ConcatV2 gemeenschappelijk hebben is dat zij een input krijgen van de tfl.div opperatie die is geconverteerd van tf.Realdiv.
Door het toevoegen van TensorFlow Flex opperaties kan het model zonder problemen geconverteerd worden naar een TFLite model.
Omdat we gebruik maken van Flex opperaties moet de TensorFlow core bibliotheek mee ge\"implementeerd worden in Android studio.

Als we het TFLite model willen implementeren in Android studio via de TensorFlow Lite Task Library dan moeten we Metadata aan het model toevoegen.
We moeten Metadata toevoegen aan het TFLite model zodat bij het uitvoeren van het model de TFLite API weet wat het input en output types zijn van het model.
Maar bij het toevoegen van Metadata aan het TFLite model krijgen we de volgende fout: Keyerror 2708.
Deze fout geeft weinig informatie, maar de oorzaak is dat de methode die de Metadat aan het model toevoegt maximaal 4 outputs verwacht.
Het geconverteerd Faster-RCNN model heeft 8 outputs.
Door het aantal outputs te reduceren tot 4 outputs kunnen we succesvol Metadata aan het model toevoegen.
Bij het uitvoeren van het het model met metadata krijgen we de volgende error: ....
Deze fout onstaat doordat tijdens het converteren naar TFLite de output vorm van [1, 300, 4] naar [1, 1, 1] verandert voor de bounding box co\"ordinaten.
Het model zal nog steeds meerdere bounding boxen voorspellen, dus het model verwacht een output buffer van [1, 300, 4].
Maar de outputbuffer die de TensorFlow Lite Task Library aanmaakt is van grootte [1, 1, 1].

We kunnen het TFLite model ook implementeren via de TensorFlow Lite Inerpreter API, hierbij is er geen Metadata nodig.
Door gebruik te maken van de TensorFlow Lite Inerpreter API moeten we zelf de input en output buffers defini\"eren.
De vereiste informatie om de correcte buffers te defini\"eren kan uit het het TFLite model gehaald worden.
Het TFLite model bevat de juiste informatie voor de inputbuffer, dus deze kan eenvoudig aangemaakt worden.
Bij het aanmaken van de outputbuffers krijgen we terug een fout dat onze outputbuffer van de verkeerde grootte is.
Dit komt doordat de output vorm volgens het TFLite model [1, 1, 1] is, maar het model levert meer resultaten.
De TensorFlow Lite Inerpreter API geeft ons de mogelijkheid om de outputbuffers aan te passen zodat deze de gewenste grootte hebben.
Op deze manier kunnen we succesvol een Faster-RCNN model uitvoeren op een mobiel apparaat.

%Bij het converteren naar een TFLite model via de concrete\_function methode kunnen we de conversie uitvoeren zonder TensorFlow flex ops.

\subsection{ONNX implementatie}
In het Faster-RCNN model wordt er gebruik gemaakt van de Round opperatie die pas beschikbaar is sinds opset versie 11.
Al de andere opperaties van het Faster-RCNN model worden ondersteund in eerdere opset versies.
De round operatie zal ervoor zorgen dat bij het converteren naar onnx formaat minstens een opset versie 11 nodig is.


%de fasterrcnn model heeft een ouput type int32 wat niet ondersteund wordt door tensorflow.
%bij het uitvoeren van het onnx model 

\begin{table}[!ht]
    \caption{Alle operaties die terug te vinden zijn in het ResNet50 model en hun compatibiliteit met andere frameworks}
\begin{tabular}{cccc}
    \hline
    Operaties & TensorFlow \textrightarrow TFLite & ONNX Opset & MACE \\
    \hline
    % All & / & 1 & / \\
    % Assert & / & / & / \\
    % BroadcastArgs & / & / & / \\
    BroadcastTo & ond & 8 & / \\
    ConcatV2 & / & 1 & ond \\
    Equal & ond & 1 & / \\
    Exp & ond & 1 & / \\
    ExpandDims & ond & 1 & / \\
    Fill & ond & 7 & / \\
    Floor & ond & 1 & ond \\
    GatherV2 & ond & 1 & / \\
    Greater & ond & 1 & / \\
    Less & ond & 1 & / \\
    LogicalAnd & ond & 1 & / \\
    Maximum & const,verw,fus & 1 & / \\
    Minimum & const,verw,fus & 1 & / \\
    NonMaxSuppressionV5 & ond & 10 & / \\
    Range & ond & 7 & / \\
    RealDiv & / & 1 & / \\
    Relu6 & const,verw,fus & 1 & ond \\
    Reshape & ond & 1 & ond \\
    ResizeBilinear & ond & 7 & / \\
    Round & ond & 11 & / \\
    SelectV2 & ond & 7 & / \\
    %Size & / & 1 & / \\
    Slice & ond & 1 & ond \\
    Softmax & ond & 1 & ond \\
    Split & ond & 1 & / \\
    Sqrt & const,verw,fus & 1 & / \\
    Square & const,verw,fus & 1 & / \\
    Squeeze & ond & 1 & / \\
    %StatelessIf & / & 1 & / \\
    Sub & const,verw,fus & 1 & / \\
    Sum & ond & 1 & / \\
    Tile & const,verw,fus & 1 & / \\
    TopKV2 & ond & 1 & / \\
    Transpose & ond & 1 & / \\
    Unpack & ond & 1 & / \\
    Where & ond & 9 & / \\
    ZerosLike & ond & 1 & / \\
    \hline
\end{tabular}
\label{tab:TFop}
\end{table}

\section{Van PyTorch naar mobiele implementatie}
Om van een PyTorch object detectie model te gaan naar een mobiele implementatie kunnen we een Pytorch of een ONNX model implementeren in Android Studio.
Het model dat we willen converteren is een voorgetrained Faster-RCNN uit de Torchvision bibliotheek.
Dit model gaan we vervolgens converteren naar een TorchScript model een ONNX model

\subsection{TorchScript implementatie}

Bij het converteren naar een Torchscript module dat verder ge\"optimaliseerd kan worden voor mobiel gebruik ondersteund PyTorch enkel jit.script voor Faster-RCNN.
Na het converteren kunnen we het model verder optimaliseren voor mobiel gebruik.
Bij het opslaan van deze ge\"optimaliseerde scriptmodule voor mobiel gebruik crasht Google Colaborate zonder een boodschap.
In plaats van het model op te slaan met de \_save\_for\_lite\_interpreter methode moeten we het model opslaan voor script gebruik.
Hiervoor moet ook de android\_pytorch bibliotheek worden ge\"implementeerd in plaats van de android\_pytorch\_lite bibliotheek.
bij het uitvoeren van het script model krijgen we een fout dat de .nms opperatie niet wordt ondersteund.
PyTorch geeft de mogelijkheid om de torchvision\_ops bibliotheek te implementeren die ervoor zorgt dat alle faster rcnn operaties worden ge\"implementeerd.
Maar na het implementeren van deze bibliotheek krijgen we de volgende error ....
We kunnen de Torchvision\_ops bibliotheek die terug te vinden is in de Github repositorie van Torchvision handmatig te implementeren in het Android studio project.
Op deze manier kunnen de Torchvision opperaties wel implementeren in Android studio.
Wel moet het PyTorch model volledig onder CPU runtime worden geconverteerd naar een TorchScript model.
anders krijgen we de volgende fout: .... %empty strided.
Op deze manier kunnen we succesvol een PyTorch Faster-RCNN model uitvoeren op een mobiel toestel.

\subsection{ONNX implementatie}
Zoals bij TensorFlow is hier ook een minimale opset versie van 11 vereist.
Al de andere opperaties van het Faster-RCNN model worden ondersteund in eerdere opset versies.
hjhjdghjdgh
