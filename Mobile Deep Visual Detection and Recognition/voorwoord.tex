Een masterproef is het afsluitstuk van zes jaar studeren.
Waarin al de theoretische en praktische kennis die ik de afgelopen jaren heb geleerd kan toepassen en combineren.
Met deze masterproef heb ik mijn eerste stappen in de onderzoekswereld genomen.
Deze laatste periode in mijn studies heb ik ervaren als een stressvol maar leerrijk semester.

De weg die ik genomen heb om dit moment te bereiken was niet een standaard traject.
Ik heb namelijk eerst een bachelor behaald in de toegepaste informatica aan de UCLL.
Om vervolgens een schakeljaar en masterjaar industri\"eel - ingenieur met specialisatie elektronica ict te volgen.
Vanwege dit traject heb ik mijn masterproef voltooid op \'e\'en semester waardoor de stressfactor aanzienlijk toenam.

Ik wil graag een aantal mensen bedanken.
Zonder hun hulp zou ik nooit tot op dit punt en eindresultaat gekomen zijn.
%zou ik nooit tot dit eindresultaat gekomen zijn.

Eerst wil ik namelijk mijn co-promotor Floris De Feyter bedanken voor zijn begeleiding van deze masterproef.
De wekelijkse feedback was altijd welkom, waardoor ik elke week met een nieuwe focus aan de masterproef kon werken.
Wanneer ik vast zat, gaf hij nieuwe idee\"en zodat ik altijd iets had waarmee ik verder kon werken.

Ook wil ik mijn promotor professor Toon Goedem\'e bedanken.
Vanwege zijn enthousiasme over deze opleiding en kennis over het onderwerp was ik tijdens de opendeurdag van campus De Nayer helemaal overtuigd om het schakeljaar te volgen.

Ik wil ook mijn ouders en zussen bedanken die mij tijdens mijn studies altijd hebben ondersteund. 
Ook wil ik hen bedanken om mij de kansen te geven die ik nodig had om dit moment te bereiken.

Ten laatste wil ik mijn vrienden bedanken om ervoor te zorgen dat ik af en toe mijn gedachten kon verzetten in deze stressvolle periode.

Thijs Vercammen